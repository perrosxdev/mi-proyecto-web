\documentclass[12pt]{article}
\usepackage[utf8]{inputenc}
\usepackage[spanish]{babel}
\usepackage{enumitem}
\usepackage{hyperref}
\usepackage{geometry}
\geometry{margin=2.5cm}

\title{Informe de Avance del Proyecto}
\author{Carlos Ulloa Vera}
\date{\today}

\begin{document}

\maketitle

\section*{Datos Generales}

\begin{itemize}
  \item \textbf{Nombre del Proyecto:} Attendify
  \item \textbf{Objetivo:} Crear un sistema web para registrar asistencia y gestionar vacaciones de empleados.
\end{itemize}

\section*{Avances Logrados}

\subsection*{1. Implementación de Contextos}
\begin{itemize}
  \item Se creó e implementó el contexto \texttt{VacationContext} para manejar los datos relacionados con las solicitudes de vacaciones.
  \item El \texttt{VacationContext} fue integrado en la pantalla de \textbf{Gestión de Vacaciones (Admin)} para manejar el estado de las solicitudes (pendiente, aprobado, rechazado).
\end{itemize}

\subsection*{2. Ajustes Visuales}
\begin{itemize}
  \item Se realizaron pequeños ajustes visuales en varias pantallas para mejorar la consistencia del diseño.
  \item Se aplicaron estilos adicionales utilizando Tailwind CSS para alinear los elementos y mejorar la experiencia de usuario.
\end{itemize}

\section*{Comparación entre PMV y PMN}
\begin{itemize}
  \item \textbf{Elementos ya implementados en el PMN:}
  \begin{itemize}
    \item La funcionalidad principal de las pantallas ya estaba implementada en el PMN:
    \begin{itemize}
      \item \textbf{Gestión de Empleados:} Agregar, editar y eliminar empleados.
      \item \textbf{Gestión de Asistencia:} Registro manual de asistencia con selección de fecha y hora.
      \item \textbf{Gestión de Vacaciones:} Filtrado y actualización del estado de solicitudes de vacaciones.
    \end{itemize}
    \item Navegación básica entre pantallas utilizando el sistema de rutas de Next.js.
    \item Uso de Tailwind CSS para estilos básicos.
    \item Componentes básicos como \texttt{Navbar} y \texttt{Modal}.
    \item Definición de contextos (\texttt{EmployeeContext}, \texttt{UserContext}).
  \end{itemize}
  \item \textbf{Avances realizados en el PMV:}
  \begin{itemize}
    \item Creación e implementación del \texttt{VacationContext} para manejar datos relacionados con las solicitudes de vacaciones.
    \item Integración del \texttt{VacationContext} en la pantalla de \textbf{Gestión de Vacaciones (Admin)}.
    \item Ajustes visuales menores en varias pantallas para mejorar la consistencia del diseño.
  \end{itemize}
\end{itemize}

\section*{Próximos Pasos}
\begin{itemize}
  \item Completar la funcionalidad de exportación de datos en formato CSV.
  \item Desplegar la aplicación en Vercel para acceso público.
  \item Documentar el proyecto en un archivo \texttt{README.md}.
\end{itemize}

\section*{Enlaces Relevantes}
\begin{itemize}
  \item \textbf{Repositorio en GitHub:} \url{https://github.com/perrosxdev/mi-proyecto-web}
  \item \item \textbf{Despliegue en línea:} \href{https://proyecto01-orpin.vercel.app/}{https://proyecto01-orpin.vercel.app}
\end{itemize}
\section{Credenciales:}
\begin{itemize}
    \item \textbf{Usuario: admin}
    \item \textbf{Contraseña: admin123}
    \item \textbf{Usuario: juancarlosbodoque}
    \item \textbf{Contraseña: password123}
    \item Se pueden crear usuarios usando el perfil de admin, actualmente el usuario se crea creando un empleado en gestion empleados. el usuario se hace uniendo el nombre e ingresando password123 como contraseña predeterminada.
  
\end{itemize}

\end{document}
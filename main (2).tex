\documentclass[12pt]{article}
\usepackage[utf8]{inputenc}
\usepackage[spanish]{babel}
\usepackage{enumitem}
\usepackage{hyperref}
\usepackage{geometry}
\geometry{margin=2.5cm}

\title{Informe de Avance del Proyecto}
\author{Carlos Ulloa Vera}
\date{}

\begin{document}

\maketitle

\section*{Datos Generales}

\begin{itemize}
  \item \textbf{Nombre original del proyecto:} Aplicación de Registro de Asistencia y Vacaciones
  \item \textbf{Nombre definitivo del proyecto:} Attendify
\end{itemize}

\section*{Objetivo del Proyecto}
\vspace{0.5em}
\noindent
El objetivo del proyecto es desarrollar un sistema web diseñado para registrar la asistencia de los empleados y administrar permisos y vacaciones de manera eficiente. Este sistema permitirá el monitoreo en tiempo real de la asistencia, ofrecerá autenticación segura para los usuarios y generará reportes automatizados que faciliten la gestión del personal.
\vspace{4em}

\section*{1. Avances Técnicos Logrados}

\begin{itemize}
  \item \textbf{¿Cuántas pantallas tiene?}
% Inicio de Sesión (Login)
% Recuperación de Contraseña
% Registro de Asistencia
% Historial de Asistencia
% Solicitud de Vacaciones
% Historial de Vacaciones
% Gestión de Vacaciones (Administrador)
% Dashboard del Administrador
% Gestión de Empleados
% Reportes
% Configuración de Perfil
% Configuración del Sistema
% Calendario de vacaciones
% Error 404
% Acceso Denegado
  
  Actualmente se han desarrollado 3 pantallas, pantalla de Inicio de sesión, pantalla de Registro y Pantalla de Vacaciones.
  \item \textbf{¿Hay navegación entre ellas?} Sí, se está usando App router de Next.js.
  \item \textbf{¿Usaste React, React Router u otra tecnología?} Se usó React para el front y se usó React Router para la navegación, tambien se usó Next.js como framework.
  \item \textbf{¿La estructura de carpetas ya está organizada?} Sí, la estructura es la recomendada para este tipo de proyecto con react y Next.js, es decir que cada pantalla debera estar en su propia carpeta (ya que cada archivo debera llamarse page.tsx) y estas se encuentran en la carpeta src/app.
  \item \textbf{Otros avances:} 
  \begin{itemize}
    \item Se creó un navbar sencillo con el fin de estilizar más el diseño de cada pantalla y ayudar con el flujo de la navegación.
  \end{itemize}
\end{itemize}

\section*{2. Dificultades Encontradas}

\begin{itemize}
  \item \textbf{Técnicos (React, despliegue, rutas...):}
  \begin{itemize}
    \item A nivel técnico, el mayor problema que encontré fue adaptarme al sistema de rutas actual. Aunque React Router es una solución de navegación válida, su funcionalidad es limitada en comparación con el sistema de rutas basado en el \textit{App Router} de Next.js, el cual permite implementaciones más completas. Sin embargo, su configuración inicial puede resultar más laboriosa.
  \end{itemize}
  
  \item \textbf{De diseño (navegación, pantallas clave...):}
  \begin{itemize}
    \item En cuanto al diseño, no encontré mayores dificultades, ya que este fue definido previamente en el primer informe del proyecto.
  \end{itemize}
  
  \item \textbf{Personales (tiempo, comprensión del proyecto, etc.):}
  \begin{itemize}
    \item A nivel personal, el mayor obstáculo fue retomar el uso de React, ya que no lo utilizaba desde hace dos años. Durante ese tiempo, la tecnología ha evolucionado considerablemente, lo cual implicó un proceso de reaprendizaje.
  \end{itemize}
\end{itemize}


\section*{3. Uso de Inteligencia Artificial}

\begin{itemize}
  \item \textbf{¿Qué preguntas fundamentales hiciste?}
  \begin{itemize}
    \item Principalmente, mis preguntas se centraron en la infraestructura actual de los proyectos con React, así como en correcciones dentro del código. Solicité ayuda para identificar errores relacionados con las nuevas formas de estructurar proyectos, especialmente en el contexto de Next.js.
  \end{itemize}

  \item \textbf{¿Qué aprendiste o cambiaste gracias a las respuestas recibidas?}
  \begin{itemize}
    \item Gracias a las respuestas obtenidas, pude reaprender conceptos fundamentales de React, incluyendo la navegación y la estructura recomendada para proyectos utilizando Next.js. Esto me permitió adaptarme nuevamente al entorno de desarrollo moderno.
  \end{itemize}

  \item \textbf{¿Qué tanto te ayudó la IA a estructurar mejor tu PMN?}
  \begin{itemize}
    \item Me ayudó bastante, ya que al principio estaba desorientado respecto a los cambios recientes en React. La IA me permitió avanzar más rápidamente sin tener que revisar toda la documentación desde cero, lo cual optimizó significativamente mi tiempo.
  \end{itemize}
\end{itemize}


\section*{5. Enlaces Técnicos}

\begin{itemize}
  \item \textbf{Repositorio en GitHub:} \href{https://github.com/perrosxdev/mi-proyecto-web}{https://github.com/usuario/repositorio}
  \item \textbf{Despliegue online (Vercel, Render, etc.):} \href{https://proyecto01-orpin.vercel.app/}{https://proyecto01-orpin.vercel.app/}
\end{itemize}
\section*{6. Acceso}
\begin{itemize}
  \item \textbf{user:} admin
  \item \textbf{password:} 1234
\end{itemize}
\end{document}
